%%%%%%%%%%%%%%%%%%%%%%acknow.tex%%%%%%%%%%%%%%%%%%%%%%%%%%%%%%%%%%%%%%%%%
% sample acknowledgement chapter
%
% Use this file as a template for your own input.
%
%%%%%%%%%%%%%%%%%%%%%%%% Springer %%%%%%%%%%%%%%%%%%%%%%%%%%

\extrachap{Acknowledgements}

%I would like to express my heartfelt gratitude to my PhD advisor, Dr. Riccardo Munaf\'o, and my supervisor, Prof. Emiliano Votta, for their guidance and unwavering support throughout this journey.

% @@@ Non credo sia questo il tipo di ringraziamento che intendono le guidelines, tanto più che sono ringraziamenti a nome di tutti gli autori, quindi sarebbe strano avere degli autori che si autoringraziano. Negli Acknowledgements di solito si citano eventuali fonti di finanziamento, oppure persone/istituzioni che non compaiono tra gli autori ma hanno permesso di sviluppare il lavoro fornendo qualche forma di aiuto esterno (ad esempio fornendo dati o consigli). 

